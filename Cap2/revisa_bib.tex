\chapter{REVISÃO BIBLIOGRÁFICA}\label{ch:rev-bibs}

\section{O protocolo mais importante em redes de computadores}\label{sec:protIP}
A internet é uma é uma rede que interconecta milhões de computadores no mundo inteiro. Para que esta comunicação ocorra, cada dispositivo conectado necessita de um endereço único, que o diferencie dos demais. Esta é a função do IP (Internet protocol). O problema é que a cada dia que se passa, mais e mais dispositivos são conectados à internet, e, por conta disto, o número de endereço de IP estão se esgotando. 

O protocolo de internet, foi definido pelo DoD (Departamento de Defesa Americano), concebido para uso em sistemas de computação interconectados através de comutação de pacotes. As primeiras redes a utilizarem IP foram a ArpaNET (Rede militar de pesquisas, financiada pela ARPA – Advanced Research Projects Agency) e a NSFNet (Rede acadêmica de pesquisa, financiada pela NSF – National Science Foundation). (Refe: http://www.jonny.eng.br/trabip/ip.html).

O IP divide cada endereço de inter-rede em uma hierarquia de dois níveis, prefixo e sufixo, o prefixo de um endereço identifica a rede a qual o dispositivo se liga, e o sufixo identifica um dispositivo específico na rede. Um endereço de IP é um número de 32 bits. No início, o endereço pertencia a uma de cinco classes, em que a classe de um endereço era determinada pelo valor dos quatro primeiros bits, onde a uma rede física que contivesse entre 257 e 65.536 hosts era atribuído o valor de prefixo B; a redes menores era atribuído um valor prefixo de classe C, e às redes maiores era atribuído o prefixo de classe A. (Refe: Livro do joão, usado nas aulas de redes)

