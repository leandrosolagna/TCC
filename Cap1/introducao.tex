\chapter{INTRODUÇÃO}\label{ch:introducao}

O crescimento contínuo e global da Internet é um dos fenômenos mais interessantes e excitantes em redes. Há algumas décadas, não se imaginava que um projeto de pesquisa envolvendo duas universidades da Califórnia – UCLA (Universidade da Califórnia, Los Angeles) e SRI (Instituto de Pesquisa Stanford), se tornaria o principal meio de comunicação no dia a dia das pessoas.

Para que essa comunicação aconteça, protocolos que ditam as regras de conexão, entre uma origem e destino, foram criados. Ou seja, possibilitando assim a comunicação entre pessoas ou máquinas, não importando a sua localização global. Entre vários protocolos que foram criados, o mais importante foi, o IP (Protocolo de Internet). O IP é responsável por dar um endereço único para cada dispositivo conectado em uma rede de computadores, permitindo assim, que as informações cheguem ao seu destino. citar: http://www.infowester.com/ip.php

A partir de 1983, a Internet começou a usar o IP versão 4 (IPv4). Isto fez com que a rede mundial de computadores crescesse em um ritmo acelerado, com vários equipamentos sendo criados e necessitando de um endereço único para funcionar normalmente. Por causa disso, cientistas e especialistas perceberam no final da década de 80, que os números de IPv4 estavam próximos de esgotarem. citar: http://ipv6.br/post/ipv6-um-desafio-tecnico-para-a-internet/

Dessa forma, começaram pesquisas para se encontrar uma solução adequada para esse problema. Nos anos 90, solução como o CIDR (Roteamento Intra Domínio Sem Classe) foi proposto, com o intuito de usar os endereços IPs de forma mais eficiente. Em tempos mais recentes, começo dos anos 2000, o NAT (Tradução de Endereço de Rede) foi proposto, mascarando IPs inválidos para que os usuários possam acessar a nuvem. Obviamente, essas soluções só foram propostas como temporárias, o grande problema ainda existia, o limite de endereços IPv4.
A solução proposta foi a atualização do protocolo, usado na época, para um outro que pudesse gerar mais endereços únicos. Foi então desenvolvido o IP versão 6 (IPv6). Nessa versão, a característica principal foi o aumento de número de endereços, passando de 4 bilhões para mais de 340 undecilhões. citar: http://ipv6.br/post/enderecamento/

Os cientistas aproveitaram essa oportunidade para mudar outras características do protocolo, algumas dessas foi a alteração do cabeçalho para um menos complexo de se entender, protocolos que auxiliam o IPv6 no seu funcionamento, ICMP (Protocolo de Controle de Mensagem da Internet), DHCP (Protocolo de Configuração Dinâmica de Host) e IPSec (Protocolo de Segurança) também foram modificados.

Assim como na versão anterior, o IPv6 usa o protocolo de controle de mensagens da Internet ICMP, que tem como responsabilidade enviar mensagens de erro indicando que o serviço, roteador ou um endereço de rede está indisponível. No IPv6, além de reportar erros no processamento de pacotes, realizar diagnósticos e enviar mensagens sobre as características da rede, o protocolo ICMPv6 assumiu também funcionalidades dos protocolos ARP (Protocolo de Resolução de Endereço), RARP (Protocolo de Resolução Reversa de Endereços) e IGMP (Protocolo Gerenciamento de Grupos da Internet). citar: http://www.teleco.com.br/tutoriais/tutorialredeipmig1/pagina_2.asp

Os protocolos ARP e RARP, em geral, são utilizados para encontrar um endereço único em um dispositivo em uma rede. O protocolo IGMP tem como função controlar os membros de um grupo de \textit{multicast}.

Com o ICMPv6 (Protocolo de Controle de Mensagem da Internet versão 6), várias mensagens são trocadas em uma rede local, sendo quatro delas fundamentais para a descoberta de hosts IPv6 vizinhos:

\begin{itemize}
\item RS (\textit{Router Solicitation}): mensagem enviada de uma estação que deseja aprender informações de um roteador de uma rede local;
\item RA (\textit{Router Advertisement}): mensagens enviadas em respostas às mensagens RS, podendo também ser usada para enviar anúncios não solicitados para toda a rede, informando da sua existência;
\item NS (\textit{Neighbor Solicitation}): mensagem enviada por uma máquina com o objetivo de encontrar o endereço MAC (Controle de Acesso à Mídia) de um endereço IPv6;
\item NA (\textit{Neighbor Advertisement}): São respostas à mensagens NS ou para informar a mudança de endereço de uma máquina.
\end{itemize}

Com o protocolo ICMP atualizado, um novo protocolo dependente do ICMP foi criado, o NDP(Protocolo Descoberta de Vizinhança). O NDP do ICMPv6 permite saber quais são as máquinas da rede, encontrar roteadores vizinhos, detectar endereços duplicados e detectar outras anomalias que possam acontecer dentro de uma rede de computadores. citar: https://tools.ietf.org/html/rfc4861

O protocolo IPv6 ainda é considerado um protocolo novo, por isso, suas vulnerabilidades ainda não foram exploradas ao máximo. citar: https://securityintelligence.com/the-importance-of-ipv6-and-the-internet-of-things/

Assim como no seu antecessor (IPv4), o IPv6 possui vulnerabilidades que podem ser exploradas através de ataque como \textit{man-in-the-middle}, cujo objetivo é o roubo de informações ou ataques que possam negar algum serviço, fazendo com que o usuário da rede não consiga utilizá-la.

Para fornecer segurança contra ataques que possam se passar por uma máquina ou um dispositivo da rede foi criado uma extensão do protocolo NDP, chamado de SEND (\textit{SEcure Neighbor Discovery}). Este protocolo propõe solução aos seguintes itens: citar: https://tools.ietf.org/html/rfc3971

\begin{itemize}
\item Criação de uma cadeia de certificados;
\item A utilização de endereços gerados criptograficamente;
\item Criação de opções para proteger todas as mensagens relativas ao NDP;
\item Prevenir ataques de reenvio de mensagens por meio de duas novas opções no NDP;
\end{itemize}

Quando se usa o IPv6, sem o apoio do SEND em todos os dispositivos da rede, há sempre o risco de lidar com problemas que envolvem o envio indevido dos RA, que são mensagens enviadas por roteadores em resposta às mensagens de requisitos ou RS, por roteadores não autorizados ou indevidamente configurados na rede. citar: http://www.teleco.com.br/tutoriais/tutorialredeipmig1/pagina_2.asp

Uma alternativa para esse problema é o protocolo RA Guard (\textit{Router Advertisement Guard}), que surgiu devido à dificuldade de implementação do SEND.

O IPv6 RA Guard monitora mensagens de RA originadas de um roteador não autorizado. Esse protocolo analisa as RA e filtra as mensagens de dispositivos não autorizados, através de uma tabela, que possui as informações dos roteadores autorizados na rede. citar: https://tools.ietf.org/html/rfc6105

Essa técnica é usada para ajudar o administrador a identificar ataques como \textit{man-in-the-middle}, um roteador da rede que não está autorizado e negação de serviço por configuração inválida. https://community.infoblox.com/t5/IPv6-Center-of-Excellence/Why-You-Must-Use-ICMPv6-Router-Advertisements-RAs/ba-p/3416


\section{JUSTIFICATIVA}\label{sec:justificativa}

Atualmente, existem métodos com o objetivo de roubar informações. Alguns destes métodos podem ser utilizados para, ao infiltrar-se, se passar por um roteador ou uma máquina na rede, visando ter o tráfego da rede passando por eles, dessa forma, obtêm-se pacotes contendo informações sigilosas.

Com o intuito de evitar esses possíveis ataques, os protocolos SEND e RA Guard foram criados para que uma ação e um alerta ocorram no momento em que se detecta uma invasão dentro de uma rede.

Nos dias atuais, é necessário saber se os dispositivos de redes domésticas vendidos no mercado suportam no mínimo um desses protocolos. 

\section{OBJETIVOS}\label{sec:objetivos}

Nesta sessão serão abordados os objetivos gerais e específicos deste trabalho. Apresentando os pontos de interesse, no qual almeja-se chegar ao fim do projeto aqui apresentado.

\subsection{Objetivo Geral}

Descobrir quais atuais equipamentos de uma rede de porte pequeno, possuem suporte aos protocolos de segurança SEND e RA Guard. Para isso, haverá a necessidade de implementar esses protocolos e testá-los nos equipamentos instalados. Por fim, uma tabela com os dispositivos testados será apresentado, nela, constará os resultados obtidos durante os testes.

\subsection{Objetivos Específicos}\label{subsec:objetivos_especificos}

\begin{itemize}
\item Pesquisar sobre os funcionamentos dos protocolos SEND e RA Guard, assim como os protocolos de IPv6 que trabalham junto a eles.
\item Montar um ambiente de testes, com equipamentos atuais, para que os protocolos pesquisados possam ser colocados em prática.
\item Inserir os protocolos de proteção no ambiente de testes, para que possam ser demonstrados.
\item Através de uma tabela comparativa, apresentar quais os dispositivos atuais que possuem suporte para as técnicas de proteção abordadas.
\end{itemize}

\section{ORGANIZAÇÃO DO TRABALHO}\label{sec:organizacao-trabalho}

Esta sessão, apresentará uma breve descrição dos capítulos presentes neste trabalho.
No capítulo 2, serão descritos os fundamentos teóricos, com a finalidade de compreender o tema proposto. Temas como funcionalidades do IPv4, IPv6, segurança, ameaças e vulnerabilidades serão mostrados

No capítulo 3, são apresentados os protocolos de segurança SEND e RA Guard. E abordado a sua funcionalidade e seus componentes.

No capítulo 4, apresentam-se as formas de implementações, equipamentos necessários para isso, equipamentos das quais serão testados e definições necessárias para o entendimento acerca do tema abordado neste trabalho.

No capítulo 5, apresentam-se os resultados obtidos com o desenvolvimento do projeto de forma geral.

No capítulo 6, a conclusão e trabalhos futuros são abordados.
